\documentclass[a4paper, 12pt]{article}
\usepackage[a4paper,top=1.5cm, bottom=1.5cm, left=1cm, right=1cm]{geometry}
\usepackage{cmap}					% поиск в PDF
\usepackage{mathtext} 				% русские буквы в формулах
\usepackage[T2A]{fontenc}			% кодировка
\usepackage[utf8]{inputenc}			% кодировка исходного текста
\usepackage[english,russian]{babel}	% локализация и переносы

\usepackage{amsmath,amssymb}
\usepackage{indentfirst}
\usepackage{longtable}
\usepackage{graphicx}
\usepackage{array}
\usepackage{float}

\usepackage{floatflt}
\usepackage{wrapfig}
%\usepackage{siunitx} % Required for alignment
\usepackage{subfigure}
\usepackage{multirow}
\usepackage{rotating}
\usepackage{caption}

\graphicspath{{.}}


\title{\begin{center}Лабораторная работа №3.2.5\end{center}
Свободные и вынужденные колебания в электрическом контуре}
\author{Балдин В. А.}
\date{\today}

\begin{document}
    \pagenumbering{gobble}
    \maketitle
    \newpage
    \pagenumbering{arabic}
    \renewcommand*{\thesubsection}{\thesection.\Alph{subsection}}

    \textbf{Цель работы:} исследование свободных и вынужденных колебаний в колебательном контуре.
    \textbf{В работе используются:} осциллограф АКТАКОМ ADS-6142H, генератор сигналов специальной формы АКИП-3409/4, магазин сопротивления МСР-60, магазин емкости Р5025, магазин индуктивности Р567 типа МИСП, соединительная коробка с шунтирующей емкостью, соединительные одножильные и коаксиальные провода.

    \section{Введение}
        \subsection*{Экспериментальная установка}
            Колебательный контур состоит из постоянной индуктивности $L$ с активным сопротивлением $RL$, переменной емкости C и сопротивления $R$. Картина колебаний напряжения на емкости наблюдается на экране двухканального осциллографа. Для возбуждения затухающих колебаний используется генератор сигналов специальной формы. Сигнал с генератора поступает через конденсатор C1 на вход колебательного контура. Данная емкость необходима чтобы выходной импеданс генератора был много меньше импеданса колебательного контура и не влиял на процессы, проходящие в контуре.

            Установка предназначена для исследования не только возбужденных, но и свободных колебаний в электрической цепи. При изучении свободно затухающих колебаний генератор специальных сигналов на вход колебательного контура подает периодические короткие импульсы, которые заряжают конденсатор C. За время между последовательными импульсами происходит разрядка конденсатора через резистор и катушку индуктивности.
            \begin{figure}[H]
                \centering
                \includegraphics{img/2.png}
            \end{figure}
            \begin{figure}[H]
                \centering
                \includegraphics{img/1.png}
            \end{figure}

        \subsection*{Теоретические сведения}

            \noindent Для RLC контура применим правило Кирхгофа:
            \begin{equation}
                RI + U_C + L\frac{dI}{dt} = 0.
            \end{equation}
            Подставив в уравнение выражение для тока через 1-ое правило Кирхгофа разделив обе части уравнения на $CL$, получим:
            \begin{equation}
                \frac{d^2U_C}{dt^2} + \frac{R}{L} \frac{dU_C}{dt} + \frac{U_C}{CL}=0
            \end{equation}
            Произведём замены $\gamma = \frac{R}{2L}$ -- коэффициент затухания, $\omega_0^2 = \frac{1}{LC}$ -- собственная круговая частота, $T_0 = \frac{2\pi}{\omega_0} = 2\pi \sqrt{LC}$ -- период собственных колебаний. Тогда уравнение примет вид:
            \begin{equation}
                \ddot{U_C} + 2 \gamma \dot{U_C} + \omega_0^2U_C = 0,
            \end{equation}
            где точкой обозначено дифференцирование по времени. Будем искать решение данного дифференциального уравнения в классе функций следующего вида:
            $$U_C(t) = U(t)e^{- \gamma t}.$$
            Получим:
            \begin{equation}
                \ddot{U} + \omega_1^2 U = 0,
            \end{equation}
            где
            \begin{equation}
                \omega_1^2 = \omega_0^2-\gamma^2
            \end{equation}
            Для случая $\gamma < \omega_0$ в силу того, что $\omega_1 > 0$, получим:
            \begin{equation}
                U_C(t) = U_0 \cdot e^{-\gamma t} \text{cos}(\omega_1 t + \varphi_0).
            \end{equation}
            Для получения фазовой траектории представим формулу в другом виде:
            \begin{equation}
                U_C(t) = e^{-\gamma t}(a \text{cos} \omega_1 t + b \text{sin} \omega_1 t),
            \end{equation}
            где $a$ и $b$ получаются по формулам:
            $$a = U_0 \text{cos} \varphi_0, \qquad b = - U_0 \text{sin} \varphi_0.$$
            В более удобном виде запишем выражения для напряжения на конденсаторе и токе через катушку:
            \begin{equation}
                U_C (t) = U_{C0} \cdot e^{-\gamma t} (\text{cos} \omega_1 t + \frac{\gamma}{\omega_1} \text{sin} \omega_1 t),
            \end{equation}
            \begin{equation}
                I(t) = C\dot{U_C}= - \frac{U_{C0}}{\rho} \frac{\omega_0}{\omega_1} e^{-\gamma t} \text{sin} \omega_1 t.
            \end{equation}
            Введём некоторые характеристики колебательного движения:
            \begin{equation}
                \tau = \frac{1}{\gamma} = \frac{2L}{R},
            \end{equation}
            где $\tau$ -- время затухания (время, за которое амплитуда колебаний уменьшается в $e$ раз).
            \begin{equation}
                \Theta = \text{ln} \frac{U_k}{U_{k+1}} = \gamma T_1 = \frac{1}{N_\tau} = \frac{1}{n} \text{ln} \frac{U_k}{U_{k+n}},
            \end{equation}
            где $\Theta$ -- логарифмический декремент затухания, $U_k$ и $U_{k+1}$ -- два последовательных максимальных отклонения величины в одну сторону, $N_\tau$ -- число полных колебаний за время затухания $\tau$.

            Теперь рассмотрим случай \textit{вынужденных колебаний} под действием внешней внешнего синусоидального источника. Для этого воспользуемся методом \textit{комплексных амплитуд} для схемы на рисунке (рис. 1):
            \begin{equation}
                \ddot{I} + 2 \gamma \dot{I} + \omega^2 I = - \varepsilon \frac{\Omega}{L} e^{i\Omega t}.
            \end{equation}
            Решая данное дифференциальное уравнение получим решение:
            \begin{equation}
                I = B\cdot e^{-\gamma t} \text{sin}(\omega t - \Theta) + \frac{\varepsilon_0 \Omega}{L \phi_0} \text{sin} (\Omega t - \varphi).
            \end{equation}
            Нетрудно видеть, что частота резонанса будет определяться формулой:
            \begin{equation}
                \omega_0 = \frac{1}{2 \pi \sqrt{LC}}.
            \end{equation}

            Способы измерения добротности $Q = \dfrac{W_0}{W_{loss,\,\tau}} = \dfrac{\pi}{\Theta}$:
            \begin{enumerate}
                \item с помощью потери амплитуды свободных колебаний:
                \begin{equation}
                    \Theta = \frac{1}{n} \text{ln}\frac{U_k}{U_{k+n}},
                \end{equation}
                \item с помощью амплитуды резонанса можно получить добротность (в координатах $U_C/U_0$, где $U_0$ -- амплитуда колебаний напряжения источника, от частоты генератора). Отсюда нетрудно определить декремент затухания $\gamma = \frac{\omega_0}{2Q}$,
                \item с помощью среза АЧХ на уровне 0.7 от максимальной амплитуды, тогда <<дисперсия>> ($\Delta \Omega$) будет численно равна коэффициенту $\gamma$, то есть $Q = \frac{\nu_0}{2 \Delta \Omega}$.
                \item с помощью нарастания амплитуд в вынужденных колебаниях:
                \begin{equation}
                    \Theta = \frac{\omega_0 n}{2\text{ln} \frac{U_0 - U_k}{U_0 - U_{k+n}}}.
                \end{equation}
                \item  с помощью формулы\begin{equation}
                    \Theta = \frac{1}{R}\sqrt{\frac{L}{C}}
                \end{equation}
            \end{enumerate}

    \section{Ход работы}

        $$
            C_0 = 1.2~нф
        $$

        \begin{table}[h]
            \centering
            \begin{tabular}{|c|c|c|}
            \hline
            $C + C_0$, нФ & $T_{exp}$, мкс & $T_{th}$, мкс \\ \hline
            2.2 & 94 & 93 \\
            3.2 & 113 & 112 \\
            4.2 & 127 & 128 \\
            5.2 & 144 & 143 \\
            6.2 & 159 & 156 \\
            7.2 & 171 & 168 \\
            8.2 & 178 & 180 \\
            9.2 & 191 & 190 \\
            10.2 & 199 & 200 \\
            \hline
            \end{tabular}
            \caption{Экспериментальные и теоретические значения периодов колебаний}
        \end{table}

        \begin{figure}[H]
            \centering
            \includegraphics[scale=0.5]{img/T_plot.png}
        \end{figure}

        \textbf{2.} Подберем и установим значение $C^*$ так, чтобы частота собственных колебаний была $\nu_0 = 6.5$ кГц. $C^* = \frac{1}{4\pi^2\nu_0^2L} \approx 6$ нФ. Рассчитаем теоретически критическое сопротивление контура $R_{cr} = 2\sqrt{\frac{L}{C}} = 8165$ Ом. Измеряем логарифмических декремент затухания по соседним максимумам при различных внешних сопротивлениях ($0,05 R_{cr} - 0,2R_{cr}$):
        \newpage
        \begin{table}[h]
            \centering
            \begin{tabular}{|c|c|c|c|c|}
                \hline
                $R_{\text{вн}}$, Ом & $R = R_{\text{вн}} + R_L$, Ом& $\theta = \ln \frac{U_k}{U_{k+1}}$ & $Q = \frac{\pi}{\theta}$ & $\sigma_{Q}$ \\ \hline
                408 ($0,05R_{cr}$) & 443 & 0,38 & 8,25 & 0,65 \\ \hline
                653,2 ($0,08R_{cr}$) & 688,2 & 0,54 & 5,81 & 0,32  \\ \hline
                898,2 ($0,11R_{cr}$) & 933,2 & 0,73 & 4,3 & 0,18 \\ \hline
                1143,1 ($0,14R_{cr}$) & 1178,1 & 0,92 & 3,42 & 0,11 \\ \hline
                1633 ($0,2R_{cr}$) & 1668 & 1,27 & 2,46 & 0,06 \\ \hline
            \end{tabular}
            \caption{Декремент затухания свободных колебаний}
        \end{table}

        Основная часть ошибки при расчете добротности данным способом --- случайная ошибка в измерении напряжения. Примем её за $2\%$. В таком случае $\sigma_{\theta} \approx 0,03$.

        Здесь мы приняли $R_L = 35$ Ом приблизительно.
        Построим график зависимости $\frac{1}{\theta^2} = f(\frac{1}{R^2})$. Так как
        \[\theta = \ln\left(\frac{U_k}{U_{k + 1}}\right) = \gamma T = \gamma \frac{2\pi}{\omega_1}\]
        \[\theta^2 = \gamma^2\frac{4\pi^2}{\omega_1^2} = \gamma^2 \frac{4\pi^2}{\omega_0^2 - \gamma^2}\]
        \[\frac{1}{\theta^2} = \frac{1}{4\pi^2}\left(\frac{\omega_0^2}{\gamma^2} - 1\right) = \frac{1}{4\pi^2}\left(\frac{4L}{CR^2} - 1\right)\]

        то зависимость должна получиться линейной:

        \[\frac{1}{\theta^2} = \frac{1}{R^2}\frac{L}{C\pi^2} - \frac{1}{4\pi^2}\]
        \begin{figure}[h]
            \centering
            \includegraphics[width = \textwidth]{img/Gr2.png}
        \end{figure}


        Прямая построена по 4-м точкам: всем, кроме последней, её убрали из рассмотрения, так как она плохо ложится на прямую. Коэффициент $k = (1,603 \pm 0,008)\cdot 10^6 \quad \varepsilon_k = 0,5 \%$.

        Найдем $R_{cr} = 2\pi \sqrt{k} = (7955 \pm 20)$ Ом. В пределах случайной погрешности не попадает в теоретически рассчитанное значение, вероятно потому, что мы не очень хорошо знаем $R_L$.

        \textbf{3.} Расчет добротности по спирали на фазовой плоскости. В помощью осциллографа получаем портрет колебаний на фазовой плоскости (в режиме XY), определяем декремент затухания по соседним пересечениям оси X. Так как измерения проводились на глаз, то оценим погрешность $\sigma_U = 0,1$ дел.

        \begin{table}[h]
            \centering
            \begin{tabular}{|c|c|c|c|c|c|}
                \hline
                R, Ом & $U_k$, дел & $U_{k + 1}$, дел & $\theta$ & $Q$ & $\sigma_{Q}$  \\ \hline
                443 & 4,4 & 3 & 0,38 & 8,20 & 0,86 \\ \hline
                1668 & 3,8 & 1 & 1,34 & 2,35 & 0,18 \\ \hline
            \end{tabular}
            \caption{Определение добротности по фазовой плоскости}
        \end{table}

        \textbf{4.} Рассчитаем теоретическое значение добротности через параметры контура
        \[Q = \frac{\pi}{\theta} = \frac{\pi}{\gamma T} = \frac{\pi}{\frac{R}{2L}\frac{2\pi}{\omega_1}} = \frac{L}{R}\omega_1 = \frac{L}{R}\sqrt{\omega_0^2 - \gamma^2} = \frac{L}{R}\sqrt{\frac{1}{LC} - \frac{R^2}{4L^2}} = \frac{1}{2}\sqrt{\frac{4L}{CR^2} - 1}\]

        При параметрах $L = 100$ мГн, $C = 6$ нФ имеем (примем погрешность $\sigma_{R_L} = 5$ Ом):

        1. $R_1 = 443$ Ом $Q_1 = 9,20$, $\sigma_{Q_1} = 0,10$

        2. $R_2 = 1668$ Ом $Q_2 = 2,40$, $\sigma_{Q_2} = 0,01$

        \textbf{5.} Измерение АЧХ и ФЧХ вынужденных колебаний.

        Установим на генераторе синусоидальный сигнал и будем наблюдать картину вынужденных колебаний. Занесем в таблицу полученные данные и построим графики в координатах $\frac{U}{U_{0}} = \frac{\nu}{\nu_0}$
        \begin{table}[h]
            \centering
            \begin{tabular}{|c|c|c|c|}
                \hline
                $\nu$, Гц & $U$, В & $\Delta t$, мкс & $\Delta \phi$, $\cdot \pi$ \\ \hline
                5700 & 37 & 72,4 & 0,83 \\ \hline
                5800 & 42,5 & 69,2 & 0,80 \\ \hline
                5900 & 49 & 67,2 & 0,79 \\ \hline
                6000 & 57 & 62,8 & 0,75 \\ \hline
                6100 & 68 & 58,8 & 0,72 \\ \hline
                6200 & 79 & 53,2 & 0,66 \\ \hline
                6300 & 90 & 47,6 & 0,60 \\ \hline
                6400 & 99 & 40,8 & 0,52 \\ \hline
                6500 & 102 & 34,4 & 0,45 \\ \hline
                6600 & 98 & 27,2 & 0,36 \\ \hline
                6700 & 90,5 & 22 & 0,29 \\ \hline
                6800 & 82,5 & 17,2 & 0,24 \\ \hline
                6900 & 74 & 14,8 & 0,20 \\ \hline
                7000 & 66,5 & 12 & 0,17 \\ \hline
                7100 & 59,5 & 10,4& 0,15 \\ \hline
                7200 & 54 & 8,8 & 0,13 \\ \hline
                7300 & 49,5 & 8 & 0,12 \\ \hline
                7400 & 46,5 & 6,4 & 0,095 \\ \hline
                7500 & 43 & 6 & 0,09 \\ \hline
                7600 & 40 & 4,4 & 0,07 \\ \hline
            \end{tabular}
            \hspace{.06\textwidth}
            \begin{tabular}{|c|c|c|c|}
                \hline
                $\nu$, Гц & $U_{max}$, В & $\Delta t$, мкс & $\Delta \phi$, $\cdot \pi$ \\ \hline
                5600 & 22,0 & 50 & 0,56 \\ \hline
                6000 & 27,6 & 40 & 0,48 \\ \hline
                6200 & 29,6 & 35,6 & 0,44 \\ \hline
                6400 & 30,8 & 31,2 & 0,40 \\ \hline
                6600 & 31,6 & 26,8 & 0,35 \\ \hline
                6800 & 31,6 & 22,8 & 0,31 \\ \hline
                7000 & 31,2 & 19,6 & 0,27 \\ \hline
                7200 & 30,4 & 16,8 & 0,24 \\ \hline
                7400 & 29,6 & 14 & 0,21 \\ \hline
                7600 & 29,5 & 12 & 0,18 \\ \hline
                8000 & 26,4 & 8,4 & 0,13 \\ \hline
            \end{tabular}
            \caption{АЧХ и ФЧХ для $R_1 = 443$ Ом и $R_2 = 1668$ Ом}

        \end{table}

        \begin{center}
            \includegraphics[width = \textwidth]{img/Gr3.png}
            \includegraphics[width = \textwidth]{img/Gr4.png}
        \end{center}
        \newpage
        \begin{table}[h!]
            \centering
            \begin{tabular}{|c|c|c|c|}
                \hline
                $R$, Ом & $\frac{2\Delta \Omega}{\omega_0}$ & $Q$ & $\sigma_{Q}$ \\ \hline
                443 & 0,12 & 8,33 & 0,69\\ \hline
                1668 & 0,43 & 2,33 & 0,16 \\ \hline
            \end{tabular}
            \caption{Определение добротности по графику АЧХ}
        \end{table}

        Определим добротность по графику АЧХ. $Q = \frac{\omega_0}{2\Delta \Omega}$, где $2\Delta \Omega$ - ширина резонансной кривой на уровне $U = \frac{U_0}{\sqrt{2}}$

        Рассчитаем добротность по ФЧХ. Для этого проведем горизонтальную линию через уровень, где наблюдается резонанс (ровно $\frac{\pi}{2}$ не наблюдается в экспериментальном резонансе). Затем отразим одну половину относительно этой прямой и измерим приблизительно ширину на расстоянии $\frac{\pi}{4}$ от резонанса (ровно в $\frac{\pi}{4}$ не сможем измерить ввиду недостатка точек и смещения резонанса).
        Примерные результаты:

        \begin{figure}[h]
            \centering
            \includegraphics[width = \textwidth]{img/Gr6.png}
            \includegraphics[width = \textwidth]{img/Gr5.png}
            \caption{Вспомогательные графики для определения погрешности по ФЧХ}
        \end{figure}

        \begin{table}[h]
        \centering
        \begin{tabular}{|c|c|c|c|}
            \hline
            $R$, Ом & $\frac{\Delta \omega}{\omega_0}$ & $Q$ & $\sigma_{Q}$ \\ \hline
            443 & 0,115 & 8,70 & 0,76 \\ \hline
            1668 & 0,415 & 2,41 & 0,17 \\ \hline
        \end{tabular}
        \caption{Определение добротности по графику ФЧХ}
        \end{table}

        Погрешности в этих методах сложно оценить. Примем погрешность в определении ширины резонансной кривой за $\sigma_{\frac{\Delta \omega}{\omega_0}} = 0,01$ в случае $R = 443$ Ом и $0,03$ в случае $R = 1668$ Ом. (для второго случая приходится немного экстраполировать).

        \begin{figure}[h]
            \centering
            \includegraphics[width = 0.75\textwidth]{img/osc.jpg}
            \caption{Картина биений при смещении частоты от резонансной}
        \end{figure}

        \newpage
        \textbf{6.} Итоговая таблица:
        \begin{table}[h]
            \centering
            \begin{tabular}{|c|c|c|c|c|c|}
                \hline
                $R$, Ом & $f(L, C, R)$ & $f(\theta)$ & Фаз. спираль & АЧХ & ФЧХ \\ \hline
                443 & $9,20 \pm 0,10$ & $8,25 \pm 0,65$& $8,20 \pm 0,86$ & $8,33 \pm 0,69$ & $8,70\pm 0,76$ \\
                & $(1\%)$ & $(8\%)$ & $(10\%)$ & $(8\%)$ & $(9\%)$  \\ \hline
                1668 & $2,40 \pm 0,01$ & $ 2,46 \pm 0,06$ & $ 2,35 \pm 0,18$ & $2,33 \pm 0,16  $ & $2,41 \pm 0,17 $ \\
                & $(0,3 \%)$ & $(2\%) $ & $(8\%) $ & $(7 \%) $ & $(7 \%) $ \\ \hline
            \end{tabular}
        \end{table}

	\section{Вывод}
	    В данной лабораторной работе мы исследовали свободные и вынужденные колебания в электрическом контуре и различными способами находили его добротность. Самый точный способ, конечно же, теоретический. Затем достаточно эффективен способ вычисления через декремент затухания. Фазовая спираль даёт высокую погрешность, поэтому это не очень надежный способ вычисления добротности. Способы вычисления через АЧХ и ФЧХ хороши, если есть специальная программа, позволяющая вычислять ширину резонансной кривой, и хорошо снятые данные (с этим тоже возникли проблемы). Несмотря на то, что при нашей оценке у $R_2 = 1668$ Ом относительная погрешность в этих опытах $7 \%$, данные были сняты некачественно и ориентироваться на них сложно.


\end{document}
